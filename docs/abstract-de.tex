\thispagestyle{empty}
\vspace*{0.2cm}

\begin{center}
    \textbf{Zusammenfassung}
\end{center}

\vspace*{0.2cm}

\noindent

Diese Arbeit stellt ein technisches Konzept f\"ur ein Zahlungssystem vor, mit dem sich Microservices automatisch f\"ur ihre zur Verf\"ugung bzw. in Anspruch genommen Dienste bezahlen k\"onnen. Dies ist motiviert durch den Bedarf einer besseren Kostentranparenz in gro\ss{}en Firmen, sowie der M\"oglichkeit dies f\"ur dezentralisierte Webservice Marktpl\"atze zu verwenden.\\

Das hier vorgestellte System basiert auf Blockchain-Technologien und fokussiert sich auf die L\"osung der Performance- und Skalierungsprobleme, die normalerweise in blockchain-basierten System auftreten. Hierf\"ur wurde ein spezialisierter State-Channel Ansatz entwickelt und eine prototypische Implementierung umgesetzt. Einerseits um ein Proof-of-Concept zu liefern, andererseits um in Tests Leistungsbenchmarkergebnisse zu sammeln. Der Prototyp beinhaltet nicht nur Server und Microservice Client Komponenten, sondern auch ein Webinterface, um den Zustand und die Historie der Daten auf der Blockchain in diesem Kontext zu analysieren.\\

Es wurde Wert drauf gelegt, die Trustlessness-, Sicherheits- und Dezentralisierungseigenschaften, die zu den gr\"o\ss{}ten Vorteilen einer Blockchain basierten L\"osung z\"ahlen, nicht zu kompromitieren. Auch wenn diese nicht f\"ur jeden Anwendungsfall relevant sind.\\

Diese Arbeit steht in engem Bezug zur allgemeinen Forschung zum Thema Blockchain Skalierung und Performanceverbesserung, zu Cloud Resourcen Management, sowie zu Blockchain basierten Maschiene-zu-Maschiene-Zahlungsystemen in anderen Anwendungsgebieten.
